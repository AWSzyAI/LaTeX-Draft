\documentclass[lang=cn,newtx,10pt,color=green,scheme=chinese]{elegantbook}

\title{时子延手稿}
\subtitle{Everything in \LaTeX{}}
\author{时子延}
\institute{认知与意识智能学习研究中心}
\date{\today}
% \version{0.01}
% \bioinfo{我的Blog}{\href{https//www.pm61.fun}{时子延的Blog}}
\bioinfo{我的Github}{\href{https://github.com/AWSzyAI}{时子延的Github}}
\extrainfo{不要因为一时的迷失而放弃目标,不要因为一时的困惑而选择堕落。}
\setcounter{tocdepth}{2} % 只显示二级目录
\logo{szy-logo.JPG}
\cover{cover.png}

\usepackage{array}
\newcommand{\ccr}[1]{\makecell{{\color{#1}\rule{1cm}{1cm}}}}


% 修改标题页的橙色带
\definecolor{customcolor}{RGB}{00,64,00}
\colorlet{coverlinecolor}{customcolor}
\usepackage{cprotect}

\addbibresource[location=local]{reference.bib} % 参考文献,不要删除

\begin{document}

\maketitle
\frontmatter

\tableofcontents

\mainmatter

\chapter{我是谁?}


\chapter{我想做的事情}

\begin{itemize}
    \item 学习EmeletBook模板撰写我的《时子延手稿》//TODO
    \item 举办2024跨年演讲
\end{itemize}


\chapter{本科篇:南京师范大学 人工智能}
\section{认知塑造}



\section{能力培养}
\section{打造自己的环境}
\section{经历体验}
\subsection{从零到一解决问题的能力}
\subsubsection{搭建一个Blog网站}
\section{积累资源}

\section{读书}
\section{笔记}
我现在用以下几种方式记笔记
\begin{itemize}
    \item \textcolor{red}{Obsidian}:全终端使用Obsidian做MarkDown笔记,用Call Out进行可折叠排版,部分内容作为时子延.io发布到Github.io上
    \item \textcolor{red}{PPT}:电脑,iPad端使用PPT记笔记
    \item \textcolor{red}{\LaTeX{}}:Mac,Windows使用\LaTeX{}记 《时子延手稿》
    \item \textcolor{red}{html}:把笔记在Obsidian中记录为html并发布到pm61.fun
    \item \textcolor{red}{Adobe illustrator}:使用Adobe illustrator制作思维导图
    \item \textcolor{red}{gitbook}:使用gitbook.io写书
    \item \textcolor{red}{Notion}:使用Notion做知识管理
\end{itemize}
\section{刷课}
\section{Blog}
\section{写作}


\section{我的ChatGPT使用报告}
\section{Github}
\section{English}
\section{Research \& Papers}

\section{项目项目经历}
\begin{itemize}
    \item 算法:蓝桥杯程序设计大赛
    \item 大创:手机全语音控制
    \item 小挑:南京聚视科技有限公司
    \item 大创:面向中小学科学教育大模型的领域知识增强方法研究与实现
    \item 网安:蓝桥杯CTF
\end{itemize}
\section{实习项目经历}

\chapter{技术栈}

\section{\LaTeX{}}
\subsection{ElementBook模板}
\TeX{} Live, tlshell 
CTeX

基于elementbook模板定制我的\LaTeX{}书籍:
\begin{lstlisting}{language=TeX}
\documentclass[cn]{elegantbook} 
\documentclass[lang=cn]{elegantbook}
\documentclass[device=pad]{elegantbook} %iPad模式,切边,放大
\end{lstlisting}

模板主题颜色
\begin{lstlisting}{language=TeX}
\documentclass[color=green]{elegantbook}
\end{lstlisting}
\begin{table}[htbp]
    \caption{ElegantBook 模板中的颜色主题\label{tab:color thm}}
    \centering
    \begin{tabular}{ccccccc}
    \toprule
      & \textcolor{structure1}{green} 
      & \textcolor{structure2}{cyan} 
      & \textcolor{structure3}{blue}
      & \textcolor{structure4}{gray} 
      & \textcolor{structure5}{black} 
      & 主要使用的环境\\
    \midrule
      structure & \ccr{structure1}
      & \ccr{structure2}
      & \ccr{structure3} 
      & \ccr{structure4} 
      & \ccr{structure5} 
      & chapter \ section \ subsection \\
      main      & \ccr{main1}
      & \ccr{main2}
      & \ccr{main3}
      & \ccr{main4}
      & \ccr{main5}
      & definition \ exercise \ problem \\
      second    & \ccr{second1}
      & \ccr{second2}
      & \ccr{second3}
      & \ccr{second4}
      & \ccr{second5}
      & theorem \ lemma \ corollary\\
      third     & \ccr{third1}
      & \ccr{third2}
      & \ccr{third3}
      & \ccr{third4}
      & \ccr{third5}
      & proposition\\
    \bottomrule
    \end{tabular}
\end{table}

How to DIY the color?
\begin{lstlisting}[tabsize=4]
    \definecolor{structurecolor}{RGB}{0,0,0}
    \definecolor{main}{RGB}{70,70,70}    
    \definecolor{second}{RGB}{115,45,2}    
    \definecolor{third}{RGB}{0,80,80}
    \end{lstlisting}

    https://www.fotor.com/cn

\subsection{数学建模模板}
\subsection{IEEEn模板}
\section{Linux - WSL}






\chapter{研究生篇:南京大学 人工智能}

\chapter{大事表:我在本科四年中做了什么?}
\section{大一}
\section{大二}


\datechange{2024/01/12}{尝试} 
\begin{change}
    \item 使用\LaTeX{}模板 elementbook来撰写《时子延手稿》
    \item 在WSL上安装texlive --all 来使用\LaTeX{}
\end{change}



\section{大三}
\section{大四}

\nocite{*} % 打印所有参考文献
\printbibliography[heading=bibintoc, title=\ebibname]
\appendix

\chapter{数学}

\end{document}